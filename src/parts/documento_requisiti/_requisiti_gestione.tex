% Copyright (c)  2019  FSC.
% Permission is granted to copy, distribute and/or modify this document
% under the terms of the GNU Free Documentation License, Version 1.3
% or any later version published by the Free Software Foundation;
% with no Invariant Sections, no Front-Cover Texts, and no Back-Cover Texts.
% A copy of the license is included in the section entitled "GNU
% Free Documentation License".

\chapter{Requisiti di gestione del progetto}
\section{Tempi e risorse}
Il rilascio della piattaforma web avverrà entro il giorno 21-02-2020, definendo 
un arco temporale di 15 giorni per lo sviluppo del prodotto commissionato.

Il budget disponibile è nullo in quanto il prodotto commissionato è un progetto 
a fine dell'esame di Programmazione per il Web nell'Università degli studi 
``Aldo Moro'' con sede a Taranto.

\section{Gruppo di progetto}
Il team di sviluppo della piattaforma web è il gruppo FSC. Il project manager è 
Alessandro Annese. Gli altri componenti del team sono Davide De Salvo, Andrea 
Esposito, Graziano Montanaro e Regina Zaccaria.

\section{Responsabilità del committente}
Il committente del progetto è il professor Giuseppe Desolda, docente del corso 
di Programmazione per il Web. Egli, dopo aver visionato i requisiti proposti 
dal team in aggiunta a quegli richiesti, ha dato la propria conferma sul 
procedere per la creazione del sistema Emotionally.

\section{Documentazione prevista}
La documentazione prevista in fase di rilascio è composta dal documento dei 
requisiti, dal documento di progettazione e da un manuale d'uso.

\section{Verifiche e convalide}
Per la fase di verifica e convalida si fa riferimento al capitolo riguardante 
tale fase del documento di progettazione. Inoltre, per poter verificare la 
buona scrittura del codice, verrà utilizzato SonarQube. Per poter verificare i 
requisiti di accessibilità verrà utilizzato un tool offerto dal committente.

\section{Ambiente di sviluppo}
Per lo sviluppo della piattaforma web si usa l'IDE PhpStorm (licenza gratuita 
per studenti), utilizzando il framework PHP Laravel (versione 5.8) come 
strumento di codifica. Per poter gestire le dipendenze di JavaScript viene 
utilizzato il package manager di Node NPM.

Per la stesura della documentazione viene utilizzato il linguaggio di markup 
LaTeX, utilizzando l'editor con licenza gratuita TeXstudio. Con lo stesso 
linguaggio di markup verranno creati i diagrammi UML.

Per la creazione dei mockup delle varie interfacce grafiche viene utilizzato il 
tool Adobe XD con licenza gratuita. 

Per la gestione del commit del progetto viene utilizzato il sistema GitHub. Per 
l'istallazione del web server portatile è utilizzata l'applicazione UwAmp.

Vengono sfruttate le API di Affectiva per poter effettuare l'analisi delle 
emozioni.
